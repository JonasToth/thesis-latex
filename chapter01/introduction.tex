\subsection{Motivation}

\begin{itemize}
    \item depth sensors increasingly used in the field and quality of sensors improve dramatically
    \item performance/compute costs of ICP high
    \item other methodologies on pointcloud registration are computationally expensive and not suitable for real time/global positioning
    \item due to ICP limitations, combining laserscan and depth sensor hard
    \item ICP requires correspondance assignment in the pointclouds and inital pose
    \item global positioning not possible with ICP
    \item feature based matching well established in computer vision community
    \item many tasks, like Place-Recognition and Loop Closure, Bag-of-Words and SLAM are implemented with these algorithms
    \item special mining environment result in bad lighting conditions for normal cameras, additional sensing with depth sensors increases accuracy
    \item potential utilization of high resolution terestrial laser scans made for mine surveying allows global positioning
\end{itemize}

\subsection{Problem Definition}

\begin{itemize}
    \item utilization of the classical feature-based computer vision algorithms depends on the ability to use these feature-detectors/descriptors on range data or a derived format
    \item the keypoint detectors need to produce stable and identifiable keypoints
    \item descriptors need to produce reliable results for matching performance
    \item secondary goal is fast range data processing for potential real-time applications
    \item this thesis proposes novel ways to transform range data in gray-scale images that provide benfitial appearance for feature algorithms.
    \item it implements the already proposed \gls{bearing-angle-image} and compares it to the novel \gls{flexion-image} it develops.
    \item both derived feature images performances under feature-detectors are compared and discussed
    \item additionally it provides high quality reference implementations for the proposed conversions in the form of library code and command line tools for batch converting existing images.
    \item both \gls{LIDAR} scans and depth image footage are analyzed
\end{itemize}

\subsection{Structure of this Thesis}

Section 2 introduces the related work on state-of-the-art algorithms for pointcloud registration and feature performance comparisons.
Section 3 builds the necessary foundational knowledge on range data sensors and the math on modeling their data.
It gives a high level introduction on keypoint detectors and descriptors and how their performance can be evaluated.
Section 4 describes the novel pipeline of range data processing starting a edge-preserving filtering to conversion to feature-images and finally feature detection.
Section 5 explains the experiments conducted and defines the evaluation metrics.
These experimental results are presented and discussed in section 6.
Finally, section 7 concludes the work and proposes further research areas to develop this new approach further.
