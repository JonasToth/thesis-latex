\newglossaryentry{bearing-angle}{%
    name={Bearing-Angle},
    description={The Bearing-Angle represents the angle between the lightray from the sensor to the depth value and the connection to a neighbouring depth value. The Bearing-Angle can be calculated with arbitrary neighbouring depth values. Both laserscans and depth images can be converted using a proper intrinsic calibration of the sensors}}

\newglossaryentry{bearing-angle-image}{%
    name={Bearing-Angle image},
    plural={Bearing-Angle images},
    description={The Bearing-Angle Image is a derived image type where each pixel encodes the Bearing-Angle for the lightray of this pixel. The angle is discretized from the range $(0, \pi)$ to the color depth of the target image. The resulting images are grayscale}}

\newglossaryentry{flexion-image}{%
    name={Flexion image},
    description={The Flexion Image is a method to convert range data.
    A scalar value is computed for each pixel of the range image, that measures the angle between different estimations of the surface normal.
    This measure is useful for obtaining high-level image features from low-level geometry information}}

\newglossaryentry{curvature}{%
    name={Curvature},
    description={Curvature is a concept from differential geometry that applies to manifolds. The principal curvatures $k_1$ and $k_2$ measure how surfaces bend at that point. Different specific types of curvature do exist}}

\newglossaryentry{mean-curvature}{%
    name={Mean curvature},
    description={Mean curvature is the sum of $k_1$ and $k_2$ and one of the main curvature types}}

\newglossaryentry{gaussian-curvature}{%
    name={Gaussian curvature},
    description={Gaussian curvature is the product of $k_1$ and $k_2$ and is used in many fields. The Gaussian curvature characterizes points on a surface by its sign and magnitude. For negative values of the Gaussian curvature, the point is said to be hyperbolic, for positive values the point is an elliptic point. Points with Gaussian curvature being zero are parabolic points}}

\newglossaryentry{multi-directional-bearing-angle-image}{%
    name={Multi-Directional-Bearing-Angle image},
    description={This image is a feature image, similar to \Glspl{bearing-angle-image}. For each pixel, the sum of each \Gls{bearing-angle} in forward and backward direction is calculated. This angle is calculated in both diagonals, horizontal and vertical direction. Finally, the maximum value is used as pixel value after scaling}}

\newglossaryentry{sfm}{
    name={Structure-from-Motion},
    description={A processing pipeline that reconstructs world geometry from processing camera footage, extracting distance information in multiple processing steps. A final global optimization of backprojection errors can be done to improve the overall quality of the resulting model}}

% \newglossaryentry{ROC}{
%    name={Receiver-Operating-Characteristics},
%    description={}
%    type=\acronymtype}

% \newglossaryentry{ToF}{
%     name={Time-of-Flight},
%     description={A common measurement principle to calculate the distance of an object by measuring the duration of a process with known traveling speed. In the context of this thesis this means the roundtrip time of electromagnetic waves being emitted, reflected by an obstacle and sensed.},
%     type=\acronymtype}

% \newglossaryentry{Laser}{
%     name={light amplification by stimulated emission of radiation},
%     description={A Laser is a light source that emitts coherent light. This technique allows to create a narrow beam of light that stays narrow of long distances. The photons have a very narrow spectrum of frequencies and the Laser can be pulsed very fast. This makes it useful for all different kinds of applications. In the context of this thesis its relevance is for range measurements.

% \newglossaryentry{LIDAR}{
%     name={Light Detection and Ranging},
%     description={A technique similar that measure the roundtrip time of photons reflected by an obstacle and calculates the distance of the object using the speed of light. It is related to RADAR (Radio Detection and Ranging) but uses a different frequency domain of electromagnetic waves. The most common light source is a \Gls{Laser}}
%     type=\acronymtype}
