\subsection{Bearing-Angle and SURF}

\begin{itemize}
    \item scaramuzza calibration of laser scan to camera\cite{scaramuzza_iros2007}
    \item conversion of laser-scan to bearing angle image
    \item manual selection of corresponding corner features in bearing angle image and color image
    \item rigid transformation calculation with error minimization

    \item Lin et.al use automatic feature extraction on bearing angle image with SURF\cite{lin_easp2017}
    \item use for pose estimation, inital pose for ICP

    \item both formulations of the bearing angle calculation incorrect
    \item scaramuzzas formulation produces sqrt of negative number
    \item lin seem to forget the sqrt of denominator, proper derivation is done in attachement
\end{itemize}
Erratum for Scaramuzza and Lin

\subsection{Rigid and Non-Rigid Pointcloud Registration}

\begin{itemize}
    \item survey of rigid transformation point cloud registrations\cite{bellekens_ambient2014}\cite{pomerleau_2015}
    \item goal is to find a transformation between two point sets that to map one pointset to the other
    \item process is called registration and used in robotics, medicine and field
    \item 6-DoF transformation is a rigid registration and is equivalent to find the common frame of reference of both point clouds
    \item non-rigid registration includes shearing and scaling but can also imply non-linear transformations
    \item non-rigid is not considered in this thesis, because the goal is to find a pose with the precondition of rigid transformation
\end{itemize}

\subsubsection{Correspondence Based Approaches}

\begin{itemize}
    \item ICP with different formulations original\cite{besl_pami1992}
    \item pointcloud feature from auto-encoder for salient regions of the pointcloud, that is matched\cite{elbaz_cvpr2017}
    \item color supported generalized ICP\cite{korn_2014}
    \item point-to-plane ICP
    \item plane-to-plane ICP
    \item generalized ICP\cite{segal_2009}
    \item point features in range data\cite{steder_robot2010}
\end{itemize}

\subsubsection{Otha}
\begin{itemize}
    \item normal distribution transformation\cite{biber_iros2003}
    \item coherent point drift\cite{myronenko_ieee2010}
\end{itemize}

\subsection{Multi-Modal Sensor Registration}

\begin{itemize}
    \item Kinect Fusion uses ICP\cite{newcombe_ismar2011}
    \item dense visual odometry uses photometric error of all image information, instead of sparse features like SIFT\cite{kerl_icra2013}, photo consistency assumption -> the brightness of a point stays constant between two camera poses.
    \item multi-cue photometric registration\cite{corte_icra2018}
    \item image-to-geometry mapping based on correlation\cite{corsini_cgf2009}
    \item laser scan registration based on panorama pictures from coregistered camera or intensity images\cite{alba_isprs2012}
    \item SIFT realistic rendering to match images with pointclouds\cite{sibbing_3dv2013}
    \item Synthetic views from dense pointcloud intensity images, matched with camera maximizing normalized mutual information\cite{wolcott_iros2014,zhao_icra2016}
    \item laser scan registration using planar patches and image data\cite{dold_iaprs2006}
    \item all are focussed on RGB-D data and do not consider dense laser scans
    \item alignment using normalized mutual information
\end{itemize}

\subsection{Feature Algorithm Comparison}

\begin{itemize}
    \item matching image with different modalities using common feature algorithms\cite{gesto-diaz_2017}
\end{itemize}
