Scaramuzza's\cite{scaramuzza_iros2007} work on register a \acrshort{LIDAR} scanner and a camera is the source of the \gls{bearing-angle-image}.
The conversion of the range data to the image format allowed manual selection of corners, edges and other salient regions in both the color image and the \gls{bearing-angle-image}.
Minimizing the reprojection error for the selected points allows to compute a relative rotation and translation between both modalities.
Lin et al.\cite{lin_easp2017} apply the \acrshort{surf}\cite{bay_eccv06} feature detector on the \glspl{bearing-angle-image}.
The keypoints detected in multiple range images allow to reconstruct the relative pose between dense pointclouds.
They additionally compare the performance of this registration to state-of-the-art \acrshort{icp} (\acrlong{icp}) algorithms.
They conclude, that the \acrshort{icp} has higher precision but execution speed is drastically improved when using the \gls{bearing-angle} based pose as initial state for the \acrshort{icp}.
To the best of the author's knowledge, no other work includes the use of \glspl{bearing-angle-image}.

\subsection{Pointcloud Registration}

The application of a feature based pointcloud registration is a novel approach to a well researched and developed area --- pointcloud registration.
Range sensors produce a set of points and one natural task is to determine the relative transformation between two such sets.
This task has many applications in photogrammetry, robotics and engineering.

\acrshort{icp}, originally formulated by Besl\cite{besl_pami1992} is a robust and easy to implement algorithm to achieve this registration.
At its core is the assignment of point correspondences between both sets, calculating a relative pose and evaluating the disparity of both pointclouds.
The mismatch the relative pose is iteratively reduced by finding better correspondences and minimizing the distance between the points of the pointclouds.
This basic approach received multiple improvements with the generalized \acrshort{icp}\cite{segal_2009,korn_2014} providing a fully probabilistic formulation.
\acrshort{icp} provides a 6 \acrlong{DoF} (\acrshort{DoF}) registration consisting of a rotation and translation.
Convergence or an upper bound of the error is not guaranteed and in general relies on a good initial estimate of the relative pose of both pointclouds rendering it useless for global registration problems without prior knowledge.

A different approach to pointcloud registration is provided by \acrlong{NDT} (\acrshort{NDT})\cite{biber_iros2003} that does not require to establish correspondences between both pointclouds.
This methods subdivides the laserscan into a grid cells and estimates the normal distribution of measuring a point in this cell on each of these cells.
Registering a second pointcloud is a matter of maximizing the likelihood that the pointcloud is measured in the reference pointcloud.
The normal distributions are differentiable and classical numerical algorithms can be used for the maximization.

Myronenko and Song proposed the \acrlong{CPD}{Coherent Point Drift} algorithm (\acrshort{CPD})\cite{myronenko_ieee2010}.
Similar to \acrshort{NDT} the algorithm utilizes probabilistic methods to achieve the registration.
A pointcloud is modelled with \acrlong{GMM} (\acrshort{GMM}) with a set of centeroids.
Alignment of two pointclouds means the maximal probability of the data points under the \acrshort{GMM}.
The centeroids of the \acrshort{GMM} are enforced to move in topological consistent fashion, hence coherent point drift.
The review articles~\cite{bellekens_ambient2014,pomerleau_2015} introduce state-of-the-art pointcloud registration algorithms both in general and for robotic applications.

Feature-like algorithms are not commonly used but some approaches have been proposed.
Elbaz et al.\cite{elbaz_cvpr2017} extract subsets of a pointcloud, analyze its main directional components using a \acrlong{PCA} and synthesize a depth map for these patches.
Those patches are encoded as a low-dimensional descriptor computed by a deep neural network based auto-encoder.
Registration of pointclouds is achieved by matching these descriptors followed by a fine-tuning step.

Steder et al.~proposed\cite{steder_robot2010} an interest point detector for range images.
The range images curvature are analyzed by computing the second derivatives after a Gaussian smoothing operation.
Salient points have a high curvature but special points like points on a line and regions of occlusion are filtered out first.
These interest points are used to calculate transformations between different range images and each potential transformation is scored by the reprojection error.

% \subsection{Multi-Modal Sensor Registration}

% \begin{itemize}
    % \item Kinect Fusion uses ICP\cite{newcombe_ismar2011}
    % \item dense visual odometry uses photometric error of all image information, instead of sparse features like SIFT\cite{kerl_icra2013}, photo consistency assumption -> the brightness of a point stays constant between two camera poses.
    % \item multi-cue photometric registration\cite{corte_icra2018}
    % \item image-to-geometry mapping based on correlation\cite{corsini_cgf2009}
    % \item laser scan registration based on panorama pictures from coregistered camera or intensity images\cite{alba_isprs2012}
    % \item SIFT realistic rendering to match images with pointclouds\cite{sibbing_3dv2013}
    % \item Synthetic views from dense pointcloud intensity images, matched with camera maximizing normalized mutual information\cite{wolcott_iros2014,zhao_icra2016}
    % \item laser scan registration using planar patches and image data\cite{dold_iaprs2006}
% \end{itemize}

\subsection{Feature Algorithm Comparison}

Keypoint detectors and feature descriptors are a well established technique in the computer vision community.
They provide a way to detect salient regions of a gray-scale image that can be consistently detected between different views of a scene.
The breakthrough development is Lowe's SIFT\cite{lowe_ijcv04} algorithm.
Early comparisons\cite{mikolajczyk_pami2005} of the SIFT to other classical detectors like the Harris corner detector\cite{harris_1988} demonstrated the strong performance of SIFT.
Since the early 2000s more work on improving upon SIFT in terms of performance, compute requirements and diverse applications lead to more detectors and descriptors, like SURF\cite{bay_eccv06}, ORB\cite{rublee_iccv11} and AKAZE\cite{alcantarilla_bmva13}.
A recent comparison between those modern, well established algorithms by Andersson and Reyna Marquez\cite{andersson_2016} still puts SIFT as a top-performer, with AKAZE\cite{alcantarilla_bmva13} yielding similar but slightly less accurate results.
ORB\cite{rublee_iccv11} on the other hand is less accurate but due to its lower computational cost still a viable option especially for mobile applications.
The use of classical feature descriptors and template based matching for different modalities, like thermal imaging, is evaluated by Gesto-Diaz et al.\cite{gesto-diaz_2017} concluding that their use within one modality results in many correct matches.
Intermodal matching is harder and depends on the modalities matched.
They note, that thermal and range images yield the worst performance regardless of the used registration technique.
The result is encouraging that algorithms like SIFT\cite{lowe_ijcv04} are not bound to color images but can be used for in scenarios with vastly different appearance.
It also motivates to search for better ways to process range data with feature based matching.
