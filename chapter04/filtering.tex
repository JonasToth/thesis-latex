\subsection{Edge-Preserving Filtering}

Filtering is an operation to reduce the impact of sensor noise from limited resolution and other random effects on measured quantities.
These deviations amplify through the feature image conversion.
Filtering reduces the influence of errors and (partly) reconstructs the underlying signal of the noisy data.
The feature images, introduced in Section~\ref{sec:feature_images}, build differentiation information from changes in the geometry of sensed objects.
Therefore, sharp changes, e.g.~edges and corners, shall be preserved by the filter requiring to edge-preserving filters.
Each applied filter uses OpenCV's\cite{opencv_library} proven and readily available implementations.

\subsubsection{Median Blur}

Median Blur, introduced to image processing by Frieden\cite{frieden_new76}, is a well established technique to reduce salt-and-pepper noise and is an effective edge-preserving filter.
The application happens on each pixel of the image.
A window of $n \times m$ pixels with $n,m \in \mathbb{N}_{2k + 1}$ is centered at the pixel.
The median of all pixels in that window is calculated and stored as the new value.
Floating point operations are not required and the filter is implementable with $\mathcal{O}(n)$\cite{huang_ieee79} time complexity.
$n$ and $m$ are the controllable parameters.
OpenCV's\cite{opencv_library} implementation additionally utilizes \acrshort{SIMD} instructions for additional processing speed.

\subsubsection{Bilateral Filter}

Bilateral filtering, introduced by Tomasi and Manduchi\cite{tomasi_iccv98}, considers two factors in the filtering process.
\emph{Spatial closeness}, similar to median filtering, and \emph{similarity} of the values are combined in the bilateral filter.
Its application is convolutional, similar to the median blur and other filters.
Each neighbouring pixels effect on the central pixel is weighted by a geometric closeness and similarity function.
Similarity can be based on a classical distance norm of photometric values or be more sophisticated like a perceived similarity.
The bilateral filter works for the general case with any function for the similarity and the closeness.
A common function for both factors is a Gaussian kernel with the Euclidean norm as distance measure.
Figure~\ref{fig:bilateral_filter} provides an example application of the bilateral filter on a single channel two-dimensional signal with the Gaussian function for \emph{closeness} and \emph{similarity}.
\begin{figure}[H]
    \begin{subfigure}[b]{0.3\linewidth}
        \includegraphics[width=\linewidth]{chapter04/img/bilateral1.png}
        \caption{}\label{fig:bilateral_1}
    \end{subfigure}
    \begin{subfigure}[b]{0.3\linewidth}
        \includegraphics[width=\linewidth]{chapter04/img/bilateral2.png}
        \caption{}\label{fig:bilateral_2}
    \end{subfigure}
    \begin{subfigure}[b]{0.3\linewidth}
        \includegraphics[width=\linewidth]{chapter04/img/bilateral3.png}
        \caption{}\label{fig:bilateral_3}
    \end{subfigure}
    \caption[Bilateral filtering visualized]{\emph{Bilateral filtering visualized.} The figures show how the bilateral filter works on a single channel step function. Figure~\ref{fig:bilateral_1} shows the original signal with random noise added. For a pixel in the center of the signal the weighting of its neighbouring pixels is computed. The weights are visualized in Figure~\ref{fig:bilateral_2}. The lower values of the step are neglected due to their lack of similarity regardless of closeness. Figure~\ref{fig:bilateral_3} shows the result of the full convolution of the filter yielding a smoothed signal without blurred edge.}\label{fig:bilateral_filter}
\end{figure}
The filter is dependant on the functions for \emph{similarity} and \emph{closeness} and the available parameters are defined through these functions.
The Gaussian case is controlled by the parameters $\sigma_d$, the standard deviation of \emph{closeness} and $\sigma_r$, the standard deviation of \emph{similarity}.
The mathematical formulation --- based on the original publication\cite{tomasi_iccv98} --- for the \emph{closeness}~$c$ in the Gaussian case of the bilateral filter is:
\begin{equation}
\begin{aligned}
    c(\xi, \vec{x})&= \exp\left(-\frac{1}{2}{\left(\frac{d(\xi, \vec{x})}{\sigma_d} \right)}^2 \right) \\
    d(\xi, \vec{x})&= d(\xi - \vec{x}) = \lnorm{\xi - \vec{x}} \text{.}
\end{aligned}
\end{equation}
This weighs any point's $\xi$ relevance from the input signal to the point $\vec{x}$ based on it's distance.
The definition of the \emph{similarity}~$s$ follows the same principle:
\begin{equation}
\begin{aligned}
    s(\xi, \vec{x})&= \exp\left(-\frac{1}{2}{\left(\frac{\theta(\mathbf{f}(\xi), \mathbf{f}(\vec{x}))}{\sigma_r}\right)}^2\right) \\
    \theta(\phi, \mathbf{f})&= \theta(\phi - \mathbf{f}) = \lnorm{\phi - \mathbf{f}}\text{.}
\end{aligned}
\end{equation}
The function $\mathbf{f}(\vec{x})$ is the value of the signal at the position $\vec{x}$ and $\theta$ weighs these values again based on the Euclidean distance of the values.
To compute the filtered value on any point $\vec{x}$, these functions need to be applied to the whole signal.
The continuous formulation is given here, but for images the filter is a discrete convolution.
\begin{equation}
\begin{aligned}
    h(\vec{x}) &= \frac{1}{k(\vec{x})} \int_{-\infty}^{\infty} \int_{-\infty}^{\infty} \mathbf{f}(\xi) c(\xi, \vec{x}) s(\mathbf{f}(\xi), \mathbf{f}(\vec{x})) d\xi \\
    \text{with the normalization} \\
    k(\vec{x}) &= \int_{-\infty}^{\infty} \int_{-\infty}^{\infty} c(\xi, \vec{x}) s(\mathbf{f}(\xi), \mathbf{f}(\vec{x})) d\xi
\end{aligned}
\end{equation}
Higher values of $\sigma_d$ and $\sigma_r$ result in stronger filtering of the result.
Again, OpenCV's\cite{opencv_library} proven and optimized implementation is utilized.
The bilateral filter is a single pass filter, but computationally intensive and in practice slower than the median blur.
