\subsection{Noise Filtering}

Commonly used a preprocessing step for other algorithms.
Measurements are usually noisy. Filtering evens out the signal
Can be normal like gaussian blur or box filter.
Can be edge preserving to keep edge features.

\subsection{Edge-Preserving Filtering}

Filtering is an operation to reduce the impact of sensor noise.
One common filter is gaussian blur or the computationally cheaper box filter.
Those filters distribute each sensory value over a local neighbourhood with different weights.
This reduces white noise but blurs edges and shapes as well.
Therefore, these filters are not edge preserving and undesirable for the use-case of this work.
As described later, preserving sharp edges is requirement to be able to extract features from the converted depth images.
Using proven implementations from OpenCV\cite{opencv_library}.

\subsubsection{Median Blur}

\begin{itemize}
    \item kernel of odd size greater then one $ksize \times ksize$
    \item replacing the center pixel with median over all pixel in that kernel
    \item computationally cheap, easy to utilize SIMD instructions
    \item edge preserving and smoothing
    \item filling single pixel errors with reasonable value of surrounding
    \item cite a paper/book
\end{itemize}

\subsubsection{Bilateral Filter}

\begin{itemize}
    \item edge preserving filter that considers both spatial proximity as well as proximity of surrounding values
    \item smooths out smaller fluctuations in local neighbourhood, but maintains sharp edges
    \item cite proper paper
\end{itemize}

