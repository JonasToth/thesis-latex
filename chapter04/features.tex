\subsection{Feature Detection and Description}
\label{sec:feature_algorithms}

Introduction to feature detection and matching.
Goals and basic principles.
Use cases like RANSAC, Bag-of-Words and stuff.

\subsubsection{Feature Description and Matching}

maybe delete?

\subsubsection{Filtering Keypoints}

Because feature detection algorithms usually produce hundreds of keypoints per HD image it is necessary to reduce the number and only choose the best keypoints.
Most common way is to filter by response and choose only the best responses.

Segmenting the image in multiple regions and keep a constant number of keypoints per region.
This increase the coverage of the image.
If transformation between two images is calculated it reduces the error from one sided focus and helps to keep track of the image when the camera moves in arbitrary direction.

Filtering by keypoint size is an option. The experiments showed showed many very small keypoints that resulted from noisy sections.
Removing these keypoints improves matching quality and chance for false positives.

\subsubsection{SIFT}

Describe approach of SIFT keypoints and descriptor.

\subsubsection{SURF}

Describe approach of SURF keypoints and descriptor.

\subsubsection{ORB}

Describe combination of ORB algorithms, Harris Corner, FAST features.
BRIEF descriptor on multiple levels.

\subsubsection{AKAZE}

Describe AKAZE and its variants.

\subsubsection{BRISK}

Describe BRISK and AGAST keypoint detector.

