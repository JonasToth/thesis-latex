\subsection{Keypoint Detector Discussion}

The presented results for the different state-of-the-art keypoint detectors imply that good performance on classical camera footage does not automatically yield good performance in the proposed feature images.
Detectors that solely work on cornerdetection, as \acrshort{orb} does with \acrshort{fast} or the Harris-Corner-Detector, have a hard time on the feature images.
The experiments on real sensor data show, that they strongly respond to missing measurements, creating a sharp black to gray edge.
Solely relying on such areas is unstable and undesireable.
Flat out banning such keypoints on the other hand might not be the right choice either.
Shadow effects and reflective properties of surfaces can produce missing range values, too.
Areas with such consistent black patches can and should be exploited as feature.

\acrshort{surf}'s results are the most surprising, especially in comparison to \acrshort{sift}.
\acrshort{surf} is designed as a faster \acrshort{sift}, using the same principles and underlying theory.
One reason for the bad results might be too many simplifications and approximations.
The use of a box-filter compared to a Gaussian in combination with the other assumptions are justified with empirical analysis on colored images.
In these cases \acrshort{surf} performs good enough, but the bigger deviation from the theoretical background and human perception compared to \acrshort{sift} seem to hurt effectivness in the novel feature images.
Afterall, \acrshort{surf} does find true correspondences in some instances.

\acrshort{sift} and \acrshort{akaze} provide the most effective keypoint detectors tested.
\acrshort{akaze}'s keypoint sizes are not as diverse as \acrshort{sift}'s, but the absolute number of keypoints is a clear advantage for example visual odometry tasks.
