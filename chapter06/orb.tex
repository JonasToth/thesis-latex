\begin{table}[H]
    {\renewcommand{\arraystretch}{1.2}%
    \setlength{\tabcolsep}{0.3em}%
\begin{tabular}{bababab}
\toprule

\rowcolor{white} \null &
\textbf{Synthetic$_{\mathbf{\mathcal{F}}}$} & \textbf{Synthetic$_{\mathbf{\mathcal{\beta}}}$} &
\textbf{Lehrpfad$_{\mathbf{\mathcal{F}}}$} & \textbf{Lehrpfad$_{\mathbf{\mathcal{\beta}}}$} &
\textbf{Office$_{\mathbf{\mathcal{F}}}$} & \textbf{Office$_{\mathbf{\mathcal{\beta}}}$} \\
\midrule

\rowcolor{lightgray}
\textbf{Keypoint Count} &
    \num{52186} & \num{46459} &
    \num{170387} & \num{162144} &
    \num{14051} & \num{12829} \\
\textbf{Correspondences} &
    \num{17798} & \num{16593} &
    \num{9157} & \num{6441} &
    \num{1921} & \num{618} \\
\rowcolor{lightgray}
\textbf{True Positives} &
    \num{10085} & \num{10203} &
    \num{2730} & \num{1335} &
    \num{1246} & \num{293} \\
\textbf{False Positives} &
    \num{11341} & \num{11559} &
    \num{51575} & \num{58681} &
    \num{3889} & \num{4498} \\
\rowcolor{lightgray}
\textbf{False Negatives} &
    \num{7713} & \num{6390} &
    \num{6427} & \num{5106} &
    \num{675} & \num{325} \\
\textbf{Precision} &
    \num{0.471} & \num{0.469} &
    \num{0.050} & \num{0.022} &
    \num{0.243} & \num{0.061} \\
\rowcolor{lightgray}
\textbf{Recall} &
    \num{0.567} & \num{0.615} &
    \num{0.298} & \num{0.207} &
    \num{0.649} & \num{0.474} \\
\textbf{Youden's index} &
    \num{0.234} & \num{0.225} &
    \num{-0.022} & \num{-0.170} &
    \num{0.322} & \num{0.099} \\
\rowcolor{lightgray}
\textbf{Accuracy} &
    \num{0.633} & \num{0.612} &
    \num{0.659} & \num{0.606} &
    \num{0.670} & \num{0.617} \\
\bottomrule
\end{tabular}

    }
    \caption{Performance indicators for the default configuration of the SIFT algorithm on the different datasets.}
\end{table}
\acrshort{orb}'s detector finds less keypoints than \acrshort{sift} for both \glspl{flexion-image} and \glspl{bearing-angle-image}.
But their size, response and location distribution (Appendix~\ref{sec:orb_stats}) indicate severe problems.
Keypoints are not detected over multiple scales and the response shows a very inconsistent pattern over the different datasets.
For \emph{Lehrpfad} the keypoints cluster in the middle and the borders of the field-of-view of the depth sensors.
Those regions contains sharp edges from feature image to black because of missing range data.
The \acrshort{fast} keypoint detector characteristics of comparing brightness values at different sampling points is to blame for the keypoints bad performance.
Similar to \acrshort{surf}'s shortcoming, it does not deal well with the smoother gray changes but requires sharp edges.
As \acrshort{orb} employs blurring and scale space, too, the effect of missing contrast between sampling points worsens in higher levels of the pyramid, yielding no keypoints there.

\begin{figure}[htp]
\begin{subfigure}[t]{0.45\linewidth}
    \includegraphics[width=\linewidth]{chapter06/results/ORB/flexion/descriptor_distances.pdf}%
    \caption{\gls{flexion-image} Descriptor Distances}
\end{subfigure}\quad
\begin{subfigure}[t]{0.45\linewidth}
    \includegraphics[width=\linewidth]{chapter06/results/ORB/bearing/descriptor_distances.pdf}%
    \caption{\gls{bearing-angle-image} Descriptors Distances}
\end{subfigure}
    \caption{Descriptor Distances for \acrshort{orb}}\label{fig:orb_descriptor_distances}
\end{figure}
\begin{figure}[htp]
\begin{subfigure}[t]{0.45\linewidth}
    \includegraphics[width=\linewidth]{chapter06/results/ORB/flexion/roc.pdf}%
    \caption{Flexion Image ROC}
\end{subfigure}\quad
\begin{subfigure}[t]{0.45\linewidth}
    \includegraphics[width=\linewidth]{chapter06/results/ORB/bearing/roc.pdf}
    \caption{Bearing-Angle Image ROC}
\end{subfigure}
    \caption{\acrshort{orb}'s performance in ROC space}\label{fig:orb_roc}
\end{figure}
The descriptor distance in Figure~\ref{fig:orb_descriptor_distances} show some separation between true positives and false positives better than \acrshort{surf} but worse than \acrshort{sift}.
In ROC space (Figure~\ref{fig:orb_roc}) these mixed result show in random performance for \emph{Lehrpfad} and OK performance for \emph{Synthetic} and \emph{Office}.
Everything considered, \acrshort{orb} is not suitable for the proposed feature images.
