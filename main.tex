\documentclass[doktyp=marbeit,fontsize=11pt]{TUBAFarbeiten}

% packages
\usepackage{amsmath}
\usepackage[ngerman]{babel}
\usepackage{blindtext}
\usepackage{caption}
\usepackage{calc}
\usepackage{cite}
\usepackage{enumitem}
%\usepackage{float} % for better picture
\usepackage[T1]{fontenc}
\usepackage{floatrow} % rows of table and pictures
\usepackage{graphicx}
\usepackage{gensymb}
\usepackage[utf8]{inputenc}
\usepackage{makeidx}
\usepackage[fleqn]{mathtools} % mathtools und links buendig machen
\usepackage{multirow} % tabellen mit mehreren zeilen pro zelle
\usepackage{subcaption}
\usepackage{subscript} % tief stellen
%\usepackage[square]{natbib}

\DeclarePairedDelimiter{\abs}{\lVert}{\rVert} % definiere absoluten betrag

% tubaf zeugs
\TUBAFFakultaet{Fakultät für Mathematik und Informatik}
\TUBAFInstitut{Institut für Informatik}
\TUBAFLehrstuhl{Lehrstuhl für virtuelle Realität und Robotik}
\TUBAFTitel{Alles}
\TUBAFBetreuer{Prof. Dr. Bernhard Jung}
\TUBAFKorrektor{Dipl-Ing. Mark Sastuba}
\TUBAFAutor[J. Toth]{Jonas Toth}
\TUBAFStudiengang{Angewandte Informatik}
\TUBAFMatrikel{57319}
%\TUBAFAnmeldeDatum[2017-02-8]{8. Februar 2017}
\TUBAFDatum[2019-09-25]{25. September 2019}

%\setcounter{tocdepth}{2}
%\setcounter{secnumdepth}{3}

\makeindex

% start the content
\begin{document}

\maketitle
\TUBAFErklaerungsseite
\tableofcontents

% \newpage
% 
% \input{verzeichnisse/bezeichnungen.tex}
% \newpage
% 
% \input{einfuehrung/content.tex}
% \newpage
% \input{grundlagen/content.tex}
% \newpage
% \input{modelle/content.tex}
% \newpage
% \input{intrinsic/content.tex}
% \newpage
% \input{extrinsic/content.tex}
% \newpage
% \input{evaluierung/content.tex}
% \newpage
% \input{fazit/content.tex}
% \newpage
% 
% \appendix
% \input{anhang/content}
% \newpage
% 
% \bibliography{verzeichnisse/literatur}
% \bibliographystyle{plain}
% 
% \listoftables
% \listoffigures
% \newpage

%\renewcommand{\indexname}{Stichwortverzeichnis}
%\addcontentsline{toc}{section}{Stichwortverzeichnis}
% \printindex

\end{document}
