\documentclass[doktyp=marbeit,fontsize=12pt,sprache=english,hausschrift=true,fleqn]{TUBAFarbeiten}

% packages
\usepackage{amsmath}
\usepackage[ngerman]{babel}
\usepackage{blindtext}
\usepackage{caption}
\usepackage{calc}
\usepackage{cite}
\usepackage{enumitem}
\usepackage[T1]{fontenc}
\usepackage{floatrow} % rows of table and pictures
\usepackage{graphicx}
\usepackage{gensymb}
\usepackage[utf8]{inputenc}
\usepackage{makeidx}
\usepackage[fleqn]{mathtools} % mathtools und links buendig machen
\usepackage{subcaption}
\usepackage{listings}

\lstset{language=C++,
    keywordstyle=\color{blue}\bfseries,
    commentstyle=\color{green},
    stringstyle=\ttfamily\color{red!50!brown},
    basicstyle=\linespread{0.8}\ttfamily,
    numbers=left,
    stepnumber=1,
    showstringspaces=false}
\lstset{literate=%
   *{0}{{{\color{red!20!violet}0}}}1
    {1}{{{\color{red!20!violet}1}}}1
    {2}{{{\color{red!20!violet}2}}}1
    {3}{{{\color{red!20!violet}3}}}1
    {4}{{{\color{red!20!violet}4}}}1
    {5}{{{\color{red!20!violet}5}}}1
    {6}{{{\color{red!20!violet}6}}}1
    {7}{{{\color{red!20!violet}7}}}1
    {8}{{{\color{red!20!violet}8}}}1
    {9}{{{\color{red!20!violet}9}}}1
}

% Images with mathcha.io 
\usepackage{tikz}
\usetikzlibrary{fadings}
\usepackage{mathdots}
\usepackage{yhmath}
\usepackage{cancel}
\usepackage{color}
\usepackage{siunitx}
\usepackage{array}
\usepackage{multirow}
\usepackage{amssymb}
\usepackage{gensymb}
\usepackage{tabularx}
\usepackage{booktabs}
% Images end


\usepackage{subscript} % tief stellen
%\usepackage[square]{natbib}
\usepackage{setspace}
\renewcommand{\baselinestretch}{1.5}

\usepackage{mathtools}
\DeclarePairedDelimiter{\ceil}{\lceil}{\rceil}
\DeclarePairedDelimiter{\floor}{\lfloor}{\rfloor}
\DeclarePairedDelimiter{\abs}{\lVert}{\rVert} % definiere absoluten betrag

\usepackage[acronym,toc]{glossaries}
\glstoctrue%

\newglossary[nlg]{symbols}{nls}{nlo}{Symbol Definition}
\makeglossaries%
\loadglsentries{glossary}
\loadglsentries{symbols}

% tubaf zeugs
\TUBAFFakultaet{Faculty of Mathematics and Computer Science}
\TUBAFInstitut{Institute for Computer Science}
\TUBAFLehrstuhl{Professorship for Virtual Reality and Multimedia}
\TUBAFTitel{Post-Processing of Depth Images and Laser Scan Data for Feature-based Pose Estimation}
\TUBAFBetreuer{Prof.\ Dr.\ Bernhard Jung}
\TUBAFKorrektor{M. Sc. Robert Lösch}
\TUBAFAutor[J. Toth]{Jonas Toth}
\TUBAFStudiengang{Master Angewandte Informatik}
\TUBAFMatrikel{57319}
% \TUBAFAnmeldeDatum[2019-09-25]{25. September 2019}
\TUBAFDatum{10. June 2020}


\setcounter{tocdepth}{3}
%\setcounter{secnumdepth}{3}

% \makeindex

\usepackage[pdftex, unicode, hidelinks,
            pdfproducer={Latex with hyperref},
            pdfcreator={pdflatex}]{hyperref}
\hypersetup{colorlinks=false,
            citecolor=black,
            filecolor=black,
            linkcolor=black,
            urlcolor=black,
            linktoc=all}

% start the content
\begin{document}

\maketitle
\TUBAFErklaerungsseite%

\pagenumbering{roman}
\tableofcontents
\newpage

\printglossary[type=\acronymtype]%
\newpage

\printglossary[type=symbols]%
\glsaddallunused[symbols]
\newpage

\pagenumbering{arabic}

\section{Introduction}

\subsection{Reasons for Importance}
\begin{itemize}
    \item multiple sensors per sensors
    \item depth sensors increasingly used in the field
    \item not-multimodal, laserscan can not be used / hardly used
    \item global positioning not easily done, because of local maxima and computational cost of for example particle based approaches
    \item performance/compute costs of ICP high
\end{itemize}

\subsection{Current State-of-the-Art}
\begin{itemize}
    \item registering depth and range data
    \item state of the art is ICP and variants, working point based or point to plane (TODO: cite ICP comparison paper)
    \item requires pointclouds of similar resolutions
    \item dependent on initial pose
\end{itemize}

\subsection{Other Approach}

\begin{itemize}
    \item Scaramuzza\cite{Scaramuzza2007} presented work on registering a camera to a laserscaner using manually selected features in the laserscan and the image
    \item prior conversion of the laserscan to an image dubbed \Glspl{bearing-angle-image}
    \item converted image shows the local geometric structure exposed to the scan

    \item Lin et.al \cite{Lin2017} applied SURF feature detection on \Glspl{bearing-angle-image}
    \item this allows feature based posed estimation with the classical workflow used in SLAM algorithms
\end{itemize}

\subsection{Improvements}

\begin{itemize}
    \item \Glspl{bearing-angle} is calculated in only one direction and can therefore not be rotation invariant
    \item this work proposes \Glspl{flexion-image} that encodes the local geometry in all directions and is rotation invariant
    \item multiple state-of-the-art keypoint detectors and feature descriptors are compared between \Glspl{flexion-image} and \Glspl{bearing-angle-image} on various datasets taken with Kinect v2 and a full Laserscan
    \item evaluation shows better performance of \Glspl{flexion-image} with regard to keypoint quality and feature description
\end{itemize}

\subsection{Structure of this Thesis}

\section{Related Work}

\subsection{Bearing-Angle and SURF}
Erratum for Scaramuzza and Lin
\subsection{Other Registrations (ICP)}
\subsection{Multi-Modal Sensor Registration}

\section{Fundamentals}

\subsection{Depth Sensors}
\subsubsection{Time-of-Flight}
\subsubsection{Structured Light}
\subsubsection{Laserscanner}

\subsection{Image formation and camera models}

\subsection{Keypoint Detection and Feature Description}
\subsection{Statistic of Binary Classifier}
\subsection{Edge-Preserving Filtering}

\section{Feature Images}

\subsection{Bearing-Angle}
\subsection{Flexion-Image}
\subsection{Discarded Approaches}
\subsubsection{Curvature}
\subsubsection{Multi-Directional Bearing Angle}

Formulas pictures and short computational evaluation.
Discussion of characteristic for Bearing Flexion.

\section{Experiments}

\subsection{Datasets}
\subsubsection{Synthetic Scene}
Blender with known trajectory?
\subsubsection{Lehrpfad Kinect}
\subsubsection{Office Kinect}
\subsubsection{Laserscans of Reiche Zeche, Wilhelm Süd}
\subsubsection{Laserscan transformed to pinhole}

\subsection{Implementation of Algorithms}
\subsection{Groundtruth Poses}
\subsection{Classification Evaluation}
\subsubsection{Approach with Backprojection and Distance Threshold}
\subsubsection{Statistical Kennziffern, what was looked at}
(Descriptors, Keypoints, Distribution, Size, Response)
\subsection{Algorithm Parameters}
\subsection{With insights, do another experiment with tuned parameters}
\subsection{Odometry with SLAM?!}

\section{Results}
\subsection{Show plain results and discuss outcome}

\section{Conclusion}

Very Sensors dependent, Time-of-flight and Laserscan gives the best quality.
SURF does not perform well, Bearing Angle gives lower response and less stability
Flexion is very nice.
FAST based stuff does not perform well, more exotic descriptors neither.
SIFT best, AKAZE very good.
Approach to transform depth image first before processing further works and gives results.

\subsection{Future Work}

\begin{itemize}
    \item BoW
    \item Laserscan with Kinect
    \item Other form of Ransac, that considers the depth information
    \item Performance Optimization, conversion is embarassingly parallel and computed on GPU
\end{itemize}

\newpage

\begin{appendix}
    % \clearpage
    \renewcommand*{\thepage}{\thesection\arabic{page}}
    \renewcommand{\thetable}{\thesection\arabic{table}}
    \renewcommand{\thefigure}{\thesection\arabic{figure}}

    % \input{anhang/content}
    \newpage

    \section{Wichtige Dinge}

    Hier stehen ganz wichtige Dinge.

    \newpage

    \addcontentsline{toc}{section}{\bibname}
    \bibliographystyle{IEEEtran}
    \bibliography{references}

    \newpage
    \addcontentsline{toc}{section}{\listtablename}\listoftables

    \newpage
    \addcontentsline{toc}{section}{\listfigurename}\listoffigures

    \newpage
    %\renewcommand{\indexname}{Stichwortverzeichnis}
    %\addcontentsline{toc}{section}{Stichwortverzeichnis}
    % \printindex
\end{appendix}
\end{document}
