\newacronym{icp}{ICP}{Iterative Closest Points}
\newacronym{sift}{SIFT}{Scale Invariant Feature Transform}
\newacronym{surf}{SURF}{Speeded-Up Robust Features}

\newglossaryentry{bearing-angle}{%
    name={Bearing-Angle},
    description={The Bearing-Angle represents the angle betwen the lightray from the sensor to the depth value and the connection to a neighbouring depth value. The Bearing Angle can be calculated to arbitrary neighbouring depth values. Both laserscans and depth images can be converted using a proper intrinsic calibration of the sensors.},
    }
\newglossaryentry{bearing-angle-image}{%
    name={Bearing-Angle Image},
    description={The Bearing-Angle Image is a derived image type where each pixel encodes the Bearing-Angle for the lightray of this pixel. The angle is discretized from the range $(0, \pi)$ to the color depth of the target image. The resulting images are grayscale.},
    type=\acronymtype}

\newglossaryentry{flexion-image}{%
    name={Flexion Image},
    description={The Flexion Image is a way to convert range data.
    For each pixel of the depth image a scalar value that measures the angle between the estimated surface normals is computed and stored.
    This measure is useful for obtaining high-level image features from low-level geometry information.},
    type=\acronymtype}
