\subsection{Keypoint Detection and Feature Description}

Technique from computer vision on color images.
Salient regions of the image are of interest to extract further information.
Can be used for triangulation, for example in stereo camera systems.
Can be used to determine trajectory of single camera.
Can be used to recognize a place or search for similar images --- document extraction.

Common features are corners, edges and more complex features that are extracted with the common algorithms described later.
Each feature has a pixel location in the image and optionally a response and a size.

Recognizing such a salient point in the image from another image that is taken from a different perspective or for example after a change in the scene requires to model the local surroundings of that spot.
This step is done by feature descriptors, sometimes called keypoint descriptors.

Such a descriptor builds a model of the region around that specific keypoint.
Such a model can be a histogram of brightness values at various distances around that point.
Optionally either the descriptor or the keypoint can be assigned a rotation.
This step is not consistently done be one or the other and sometimes saved completly, because not rotation is expected to happen and the computational cost can be saved.
The model is then stored as a vector.
This vector can be of type real or contains boolean values, too.
The length varies from algorithm to algorithm, but 64 elements or more are common dimensions.
Descriptors that result in booleans can for example sample the environment of a keypoint in a predefined order and just store if the brightness is higher or lower then the previous sample.

The last and final step is matching descriptors, for simplicity reasons the case of two consecutive images in a video stream is choosen to demonstrate the principle workings.
The simplest method is to calculate the distance of each descriptor of the first image with each descriptor of the second image.
Choosing the right norm is crucial for both accuracy and speed of this step.
Dense real descriptors usually choose the L2 or L1 norm resulting in higher computational cost due to the floating point operations.
Binary descriptors use the Hamming Norm which is very cheap to compute.
Both descriptors are XOR'ed together and the number of set bits of the result is the distance of these descriptors.

Conceptually, descriptors with minimal distance correspond to each other.

Further heuristics should be applied to reduce the number of false positives and to improve the results.
The first heuristic is called cross checking. Descriptors correspond only if the distance of the descriptors is minimal in both directions.
That means that descriptor 1 has no closer descriptor in image 2, and the matched descriptor from image 2 has no closer descriptor in image 1 then descriptor 1.

The second commonly used heuristic is the lowe ratio check\cite{lowe_ijcv2004} that requires that the second best match for a descriptor must succeed a certain distance ratio.

Another simple heuristic is a maximum accepted descriptor distance.
This is different for descriptor type, -size and matching norm and requires empirical data for a given use case.
Therefore, it is not simple to define which value to choose but requires careful consideration of the operational environment and parameters in use.

More complicated preprocessing can be done to reduce the number of descriptors to match.
Some keypoints can be discarded to achieve a good distribution over the image and avoid clusters of keypoints.
Keypoints can be discarded on the response or size.
Filtering by response is commonly done when selecting only e.g. 300 keypoints.
If apriori information of the approximate motion of the camera is known matches can be ranked based on the expected displacement of the keypoint given the known motion.

In information retrieval scenarios, e.g.~a search engine for similar images, further processing is necessary.
This includes clustering of descriptors and only storing a representative as well as dimension reduction and hierarchical data structures to store the descriptors in to make the matching of a vast number of descriptor feasable.

