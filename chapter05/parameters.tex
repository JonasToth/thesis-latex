\subsection{Parameter Search}

As first step the unfiltered depth images are used for the conversions.
The keypoint count, distribution, response and size are analyzed.
Different algorithm and filter configurations are analyzed and compared to this baseline.
Notable differences receive further attention and analysis.

The effect of different filters for the depth image is tested by running the median blur and bilateral filter on the raw range data, effectively creating a new depth image source.
Each of those data sources is analyzed with the same settings for each algorithm.
The total number of configurations is the cartesian product of the data sources (Table~\ref{tab:filter}) with each algorithm configuration (Tables~\ref{tab:sift_akaze} and~\ref{tab:surf_orb}).
Only the \emph{Synthetic} dataset is not filtered as its range values have no measurement error.
\begin{table}[H]
    {\renewcommand{\arraystretch}{1.3}%
    \setlength{\tabcolsep}{0.3em}%
    \footnotesize
    \begin{tabular}{ccccc}
    \toprule
    \null & \textbf{Median Blur} & \textbf{Bilateral Filter} \\
    \midrule
    \textbf{Parameters}
        & $w = 5 \times 5px$
        & $\sigma_{color} = 25$ \\
    \null & \null & $\sigma_{space} = 7$ \\
    \bottomrule
    \end{tabular}
    }
    \caption{The parameters for the filters applied to the sensor data.}\label{tab:filter}
\end{table}
\begin{table}[H]
    {\renewcommand{\arraystretch}{1.2}%
    \setlength{\tabcolsep}{0.5em}%
    \footnotesize
    \begin{tabular}{rcc}
    \toprule
    \textbf{Name}   & \textbf{\acrshort{sift} Parameters} & \textbf{\acrshort{akaze} Parameters} \\
    \midrule
        Default     & \texttt{keypoint size > 5px}        & \texttt{descriptor: \acrshort{mldb}} \\
        \null       & \texttt{contrast threshold = 0.04}  & \texttt{diffusity: PM\_G2} \\
        \null       & \texttt{$\sigma$ = 1.6}             & \null \\
        Best only   & \texttt{best 400 keypoints}         & \texttt{best 400 keypoints} \\
        Variation 1 & \texttt{contrast threshold = 0.02}  & \texttt{descriptor: KAZE} \\
        Variation 2 & \texttt{contrast threshold = 0.08}  & \texttt{diffusity: PM\_G1} \\
        Variation 3 & \texttt{$\sigma$ = 1.0}             & \texttt{diffusity: WEICKERT} \\
        Variation 4 & \texttt{$\sigma$ = 2.2}             & \texttt{diffusity: CHARBONNIER} \\
    \bottomrule
    \end{tabular}
    }
    \caption[Parameters evaluated for \acrshort{sift} and \acrshort{akaze}]{\emph{Parameters evaluated for \acrshort{sift} and \acrshort{akaze}.} \emph{Default} is used as basis for all configurations and only differences are presented. The keypoints of \acrshort{sift} are filtered by keypoint size, because many small keypoints were detected on artifacts of sensor noise.}\label{tab:sift_akaze}
\end{table}
\begin{table}[H]
    {\renewcommand{\arraystretch}{1.2}%
    \setlength{\tabcolsep}{0.5em}%
    \footnotesize
    \begin{tabular}{rcrc}
    \toprule
    \textbf{Name} & \textbf{\acrshort{surf} Parameters} & \textbf{Name}  & \textbf{\acrshort{orb} Parameters} \\
    \midrule
        Default   & \texttt{octaves = 4}                & Default        & \texttt{Score Metric: FAST} \\
        \null     & \texttt{layers per octave = 1}      & \null          & \null \\
        Best Only & \texttt{octaves = 1}                & Default Harris & \texttt{Score Metric: Harris} \\
        \null     & \texttt{layers per octave = 1}      & \null          & \null \\
        \null     & \texttt{best 400 keypoints}         & \null          & \null \\
    \bottomrule
    \end{tabular}
    }
    \caption[Parameters evaluated for \acrshort{orb} and \acrshort{surf}]{\emph{Parameters evaluated for \acrshort{orb} and \acrshort{surf}.} Both \acrshort{orb} and \acrshort{surf} are only analyzed with two configurations. The algorithms showed systematic issues when applied to the novel feature images, that are discussed later.}\label{tab:surf_orb}
\end{table}
