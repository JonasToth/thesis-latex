\subsection{Metrics}

The measure for accuracy for each algorithm is developed in this section.
To determine the performance of the feature-based approach the matching of keypoints is framed in terms of a binary classifier.
For a keypoint correspondence between two images the classification task is to determine if those keypoint correspond to the same point in the real world or not.
This evaluation is done for \Glspl{bearing-angle-image} and \Glspl{flexion-image}.
Only two consecutive images are matched in the pinhole datasets. 
The author does not have knowledge of other approaches to evaluate feature detector and descriptors in this manner.

The following subsections describe each component of this evaluation pipeline in more detail.

\subsubsection{Groundtruth Poses}

The dataset \emph{lehrpfad} provides groundthruth poses from \gls{sfm}.
Due to the global optimization of the \gls{sfm} algorithm these poses do not match with the depth values of the depth images.
Therefore, each pose is refined with an ICP algorithm, namely OpenCV's \emph{FASTICPOdometry} that is based on KinectFusion\cite{newcombe_ismar2011}.

The other pinhole based datasets use only the ICP to provide the groundthrouth pose.
Manual inspection of the results and careful evaluation of the reprojection error validate the correctness of these poses.

\subsubsection{Approach with Backprojection and Distance Threshold}

To measure matching performance, each keypoint of two frames is classified of having a corresponding keypoint in the other frame.
This is done by projecting each keypoints camera coordinates of the previous frame into the current frame using the groundthruth pose.
The new camera coordinates in the current frame are then projected into the image with the intrinsic of the sensor.

This gives two sets of keypoints in the current frame, the keypoints detected in this frame and the keypoints detected in the previous frame, seen from this frames pose.

% TODO: Mathematik dafür einfügen

Corresponding keypoints have a small pixel distance and not corresponding keypoints a bigger pixel distance.
\emph{Small} is a relative value, but the experiments use 2 pixel as threshold.

To classify the result of the descriptor matching each keypoint needs to be analyzed in the following order.
The result are four sets \emph{true-positive}, \emph{false-positive}, \emph{true-negative} and \emph{false-negative}.
The union of these sets are all detected keypoints.

\begin{enumerate}
    \item each matched keypoint correspondence is partitioned into true positive and false positive, using the distance threshold as deciding factor
    \item false negatives are searched in the remaining query-indices, using the same distance threshold
    \item every unhandled keypoint is a \emph{true-negative} as there are no corresponding keypoints left
\end{enumerate}

The partitioning of the keypoints is the foundation for the rest of the evaluation.

\subsubsection{Histograms and Summary Statistics}

To understand the characteristics of a quantity a common approach is to create a histogram.
The histograms counts frequencies for a specific range of data.

Summary statistics are used to reduce a statistical distribution to a few representative values.
Different measures are used, such as the measure of location, distribution and shape.
Those values help to understand and compare different setups and configurations.

Both histograms and its summary statistics are created for the keypoint size, keypoint response, descriptor distance between all keypoints and both descriptor distance for \emph{true-positives} and \emph{false-positives}.

The used measures are
\begin{itemize}
    \item $\min$ and $\max$
    \item median and arithmetic mean
    \item variance and standard deviation
    \item skeweness
\end{itemize}

\subsubsection{Classification Evaluation}

The analysis of the keypoint and descriptor characteristics give already some insight into the algorithm performance but are not suitable for a comparison between different algorithms and do not give insight into potential trade-offs.
For this task the quality of the decisions the keypoint matching algorithm is required.
This is a common task in different scientific disciplines when using the classification formualation from above.
For a broad picture this evaluation uses multiple performance numbers.

\begin{itemize}
    \item precision
    \item recall or sensitivity
    \item fallout or false alarm rate
    \item accuracy or rand-index, as measure of correct decisions of the system
    \item youden-index as measure of informedness of the classifier
\end{itemize}

Each of these single measures gives insight into one aspect of decision making performance.
They do not give a systematic way to judge tradeoffs and no easy way to compare different configurations visually.

For this task, each configuration of an algorithm is plotted in \gls{ROC}-Space as a single point.
These plots make all provided configurations comparable and provide a simple way to judge the performance of a configuration.
