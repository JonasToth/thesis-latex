\newcommand{\colvecxx}[2]{\left(\begin{array}{c} #1 \\ #2 \end{array}\right)}
\newcommand{\colvecxxx}[3]{\left(\begin{array}{c} #1 \\ #2 \\ #3 \end{array}\right)}
\newcommand{\colvech}[3]{\left(\begin{array}{c} #1 \\ #2 \\ #3 \\ 1\end{array}\right)}

\newcommand{\rowvecxx}[2]{\left(\begin{array}{cc} #1 & #2 \end{array}\right)^T}
\newcommand{\rowvecxxx}[3]{\left(\begin{array}{ccc} #1 & #2 & #3 \end{array}\right)^T}

\newcommand{\lnorm}[1]{\lVert#1\rVert}

\DeclareMathOperator{\arctantwo}{arctan2}

\DeclarePairedDelimiter{\ceil}{\lceil}{\rceil}
\DeclarePairedDelimiter{\floor}{\lfloor}{\rfloor}
\DeclarePairedDelimiter{\abs}{\lVert}{\rVert} % definiere absoluten betrag
