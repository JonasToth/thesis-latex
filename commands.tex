\newcommand{\colvecxx}[2]{\left(\begin{array}{c} #1 \\ #2 \end{array}\right)}
\newcommand{\colvecxxx}[3]{\left(\begin{array}{c} #1 \\ #2 \\ #3 \end{array}\right)}
\newcommand{\colvech}[3]{\left(\begin{array}{c} #1 \\ #2 \\ #3 \\ 1\end{array}\right)}

\newcommand{\rowvecxx}[2]{\left(\begin{array}{cc} #1 & #2 \end{array}\right)^T}
\newcommand{\rowvecxxx}[3]{\left(\begin{array}{ccc} #1 & #2 & #3 \end{array}\right)^T}

\newcommand{\lnorm}[1]{\lVert#1\rVert}

\DeclareMathOperator{\arctantwo}{arctan2}

\DeclarePairedDelimiter{\ceil}{\lceil}{\rceil}
\DeclarePairedDelimiter{\floor}{\lfloor}{\rfloor}
\DeclarePairedDelimiter{\abs}{\lVert}{\rVert} % definiere absoluten betrag

% https://tex.stackexchange.com/questions/9466/color-underline-a-formula
\def\mathunderline#1#2{\color{#1}\underline{{\color{black}#2}}\color{black}}

% https://tex.stackexchange.com/questions/6850/table-and-figure-side-by-side-with-independent-captions
% Table float box with bottom caption, box width adjusted to content
% \newfloatcommand{capbtabbox}{table}[][\FBwidth]
% https://tex.stackexchange.com/questions/101980/table-and-figure-side-by-side-with-table-caption-above-figure-caption-below
% \newfloatcommand{capbtabbox}{table}[\captop][\FBwidth]
\newfloatcommand{btabbox}{table}[][\FBwidth]

\DeclareUnicodeCharacter{2718}{\nej}

