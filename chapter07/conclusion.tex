This thesis evaluated the use of state-of-the-art computer vision feature detection and description algorithms to find salient points in a depth image.
Instead of direct application of the algorithm to the depth image, it is transformed into a derived image constructed from local geometric properties of the depth values.
The approach is introduced by Scaramuzza et al.\cite{scaramuzza_iros2007} with their \gls{bearing-angle-image} and the usage of \acrshort{surf} to detect keypoints on those by Lin et al.\cite{lin_easp2017}.
To overcome the \gls{bearing-angle-image}'s limitation, the lack of rotation invariance, this work proposed four additional feature image conversions.
Only the \gls{flexion-image} is judged as viable.
The performance of \acrshort{sift}, \acrshort{surf}, \acrshort{orb} and \acrshort{akaze} were evaluated for both the \gls{bearing-angle} and the \gls{flexion-image}.
Surprisingly, the evaluation showed that \acrshort{surf} lacks descriptive power to achieve good recognition performance.
\acrshort{orb}'s corner-based keypoint detector failed to produce stable keypoints.
\acrshort{sift} and \acrshort{akaze} were the only algorithms achieving acceptable detection and recognition performance.
The findings of this thesis --- the novel \gls{flexion-image} and its performance for \acrshort{sift} and \acrshort{akaze} --- may be the foundation for a descriptor for dense, structured pointcloud data, a problem not solved satisfactory yet.

The proposed conversion has been tested with a Kinectv2 depth sensor and the Riegl Z300 \acrshort{LIDAR} scanner.
Preprocessing the depth data with edge-preserving filtering was part of the evaluation and has proven to be useful for the Riegl Z300, but unnecessary for the Kinectv2 sensor.
Applying a median blur is enough for consistent smooth results and a more advanced bilateral filtering was not beneficial.

The implementation of the evaluation software and conversion code was done with state-of-the-art software engineering principles and is of high quality.
It can be used for follow up research and various binary artifacts are produced that run on any Linux system and can be compiled from source if required.
