This thesis evaluated the use of state-of-the-art computer vision feature detection and description algorithm to find salient points in a range image.
Instead of direct application of the algorithm to the range image, it is transformed into a derived image constructed from local geometric properties of the range values.
The approach is introduced by Scaramuzza et al.\cite{scaramuzza_iros2007} with their \gls{bearing-angle-image} and the usage of \acrshort{surf} to detect keypoints on those by Lin et al.\cite{lin_easp2017}.
To overcome the \gls{bearing-angle-image}'s limitation, the lack of rotation invariance, the thesis proposed four additional feature image conversions.
Only the \gls{flexion-image} is judged as viable.
The performance of \acrshort{sift}, \acrshort{surf}, \acrshort{orb} and \acrshort{akaze} were evaluated for both the \gls{bearing-angle-image} and the \gls{flexion-image}.
Surprisingly, the evaluation showed that \acrshort{surf} lacks descriptive power to achieve good recognition performance.
\acrshort{orb}'s corner-based keypoint detector failed to produce stable keypoints.
\acrshort{sift} and \acrshort{akaze} were the only algorithms achieving acceptable detection and recognition performance.
The findings of this thesis --- the novel \gls{flexion-image} and its performance for \acrshort{sift} and \acrshort{akaze} --- may be the foundation for a descriptor for dense, structured pointcloud data, a problem not solved satisfactory yet.
